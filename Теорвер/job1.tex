\documentclass[a4paper, oneside]{book}

\usepackage[utf8]{inputenc}
\usepackage[T1]{fontenc}
\usepackage[T2A]{fontenc}
\usepackage{amsmath}
\usepackage[hmargin={2cm, 2cm}, vmargin={2cm, 2cm}]{geometry}
\usepackage{amssymb}
\usepackage{xcolor}
\usepackage{epstopdf}
\usepackage{titlesec}
\usepackage{indentfirst}

\usepackage{enumitem}

\usepackage{fancyhdr}

\usepackage[russian]{babel}
\usepackage{mathrsfs}
\usepackage{textcomp}
\usepackage{wrapfig}
\usepackage{float}

%biblatex
\bibliographystyle{gost-numeric.bbx}
\usepackage{csquotes}
\usepackage[backend=biber]{biblatex}

\begin{document}
КЕК
\begin{enumerate}
\item{Задание 1.}
10 гостей путём жеребьёвки занимают места в ряду из 10 стульев. Найти вероятность того, что два конкретных лица А и В не окажутся рядом.

Вычислим общее количество способов посадки двух гостей. Оно равно $ A_{10}^2 = 90 $ (используется число размещений, так как нам важен порядок
гостей).

Далее посчитаем, в скольких случаях гости окажутся рядом. Если гость А сидит на $\overline{2..9}$ месте, то гость В может сесть с двух сторон от него. Таким образом получает $(9 - 2 + 1) \cdot 2 = 16$ возможных случаев. Кроме того, остались неучтёнными случаи, когда гость А сидит на крайних местах. То есть нужно добавить ещё два случая. Итого в 18 случаях гости будут сидеть рядом.

Воспользовавшись классическим определением вероятности, получаем, что вероятность того, что гости не окажутся рядом, равна $\frac{90-18}{90}=0.8$

\item{Задание 2}. Вероятность безотказной работы каждорго элемента в течение времени T равна p. Элементы работают независимо и включены в цепь по приведённой схеме. Пусть событие $A_i$ означает безотказную работу за время T элемента с номером i, а событие B безотказную работу цепи.

Требуется:
\begin{enumerate}
\item{} Написать формулу, выражающую событие В через все события $A_i$.

На участках последовательного соединения вероятности корректной работы будут умножаться, так как корректная работа слелующего участка цепи, по которому течёт ток, возможна лишь только при корректной работе предыдущего.

На участках параллельного соединения же вероятности корректной работы будут складываться, так как для корректной работы цепи достаточно работы хотя бы одной ветви параллельного соединения - тогда ток потечёт.

Согласно схеме, имеем вероятность события B:
$$P(B) = P(A_1) \cdot (P(A_{23}) + P(A_{45}) + P(A_{67})) \cdot P(A_8) = $$
$$P(A_1) \cdot (P(A_2) \cdot P(A_3) + P(A_4) \cdot P(A_5) + P(A_6) \cdot P(A_7)) \cdot P(A_8)$$

\item{} Вычислить вероятность события В.
$$P(B) = p \cdot (3\cdot p^2) \cdot p = 3p^4$$
\item{} Вычислить $P(B)$ при $p=\frac{1}{2}$
$$P(B) = \frac{3}{16}$$
\end{enumerate}

\item Сообщение состоит их сигналов "1"\ и "0". Свойства помех таковы, что искажаются в среднем 5\% сигналов "0"\ и 3\% сигналов "1". При искажении вместо сигнала "0"\  принимается сигнал "1"\ и наоборот. Известно, что среди передаваемых сигналов "0"\  и "1"\ встречаются в соотношении 3:2. Найти вероятность того, что:
\begin{enumerate}
\item{} Отправленный сигнал будет принят как 1.

Пусть событие А -- отсыл сигнала "1".

Пусть событие В -- отсыл сигнала "0".

Пусть событие С -- приём сигнала "1".

Тогда $P(C) = P(A) \cdot (1 - 0.03) + P(B) \cdot 0.05 = \frac{3}{3 + 2} \cdot 0.97 + \frac{2}{3 +  2} \cdot 0.05 = 0.602$.

Вероятность события А и В - это частота посылки соответствующего сигнала.
Единицу можно принять в случае, если нам послали 1 и помеха не наложилась либо если нам послали 0, но на него легла помеха. Между вероятностями этих событий стоит знак сложения, так как события независимы и, следовательно, их вероятности складываются.


\item{} Отправленный сигнал будет принят как 0.

Пусть событие D -- приём сигнала "0".

Тогда по тому же принципу:
$P(D) = P(A) \cdot 0.03 + P(B) \cdot (1 - 0.05) = \frac{3}{3 + 2} \cdot 0.03 + \frac{2}{3 +  2} \cdot 0.95 = 0.398$.

\textit{Примечание}: можно было посчитать вероятность события D как инвертированную вероятность события С. Проведённый прямой подсчёт доказыает, что выведенная формула верна: сумма событий действительно даёт достоверное событие -- то есть здравый смысл здесь не нарушен.
\end{enumerate}

\end{enumerate}
\end{document}