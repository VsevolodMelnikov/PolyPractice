\documentclass[a4paper, oneside]{book}

\usepackage[utf8]{inputenc}
\usepackage[T1]{fontenc}
\usepackage[T2A]{fontenc}
\usepackage{amsmath}
\usepackage[hmargin={2cm, 2cm}, vmargin={2cm, 2cm}]{geometry}
\usepackage{amssymb}
\usepackage{xcolor}
\usepackage{epstopdf}
\usepackage{titlesec}
\usepackage{indentfirst}

\usepackage{enumitem}

\usepackage{fancyhdr}

\usepackage[russian]{babel}
\usepackage{mathrsfs}
\usepackage{textcomp}
\usepackage{wrapfig}
\usepackage{float}

%biblatex
\bibliographystyle{gost-numeric.bbx}
\usepackage{csquotes}
\usepackage[backend=biber]{biblatex}

\addbibresource{literature.bib}
\begin{document}
\thispagestyle{empty}

%\begin{center}
%\topskip=8cm
%\end{center}
%\large{}
%\vspace{\stretch{1}}
%\begin{flushright}
%\end{flushright}

\title{Курсовая работа по теме: \\ Генетические алгоритмы}
\author{Выполнили:\\студенты группы 3630102/70201\\Дамаскинский К.\\Колесник В.\\Пестряков Д.\\Рыженко В.}
\maketitle

\pagestyle{fancy}
\fancyhf{}
\fancyfoot[C]{\thepage}
\renewcommand{\headrulewidth}{0pt}
\renewcommand{\footrulewidth}{0pt}
\tableofcontents
\pagebreak

\chapter{Введение}
Живая природа. Сложный и в то же время удивительный механизм. Разнообразие путей её развития во все времена будоражило умы человека.
Фрэнсис Бэкон, Карл Линней, Жан Батист Ламарк и, конечно же, Чарльз Дарвин. Краткий список многочисленных имён, стоявших у истоков эволюционной теории. Труды Дарвина до сих пор остаются актуальными и имеют колоссальный вес в современной науке. Хотя все начиналось с обычного наблюдения за живой природой.
Хотя есть список имен, которые не приходят на ум при слове "эволюция", но которые должны пополнить список выше. Леонардо да Винчи, Жорж де Местраль, Ладзаро Спалланцани. Какое отношение они имеют к изучению развития живой природы? Посредственно как теоретики-эволюционисты никакого. Но именно эти великие изобретатели смогли разглядеть потенциал природы в качестве самого лучшего изобретателя. Леонардо да Винчи спроектировал свой летательный аппарат по образу и подобию обыкновенной птицы. Идея не получила широкой реализации в прямом виде, но, в конечном итоге, мы имеем современные самолеты. Справедливости ради стоит отметить, что Жорж де Местраль изобрел современные застежки-лепучки, взяв за основу семена репейника, а Ладзаро Спалланцани развил теорию эхолокации, изучая обычных летучих мышей.

Как можно заметить, применение в технических устройствах и системах принципов организации, свойств, функций и структур живой природы, именуемое биомимикрией, увенчалось технологическими прорывами. И это вполне объяснимое явление. У живой природы были миллиарды лет и бесчисленное количество испытуемых для экспериментов. Поэтому всё ныне сохранившееся в биосфере имеет место для существования в данных условиях. Ведь виды формировались, развивались и в конечном итоге неизбежно вымирали, давая возможности для развития особей менее приспособленных в старых условиях, но идеально подходящих для новых.
Развитие вычислительных систем позволило человечеству моделировать и изучать эволюционные процессы в живой природе, сократив время ожидания получения конечного результата с нескольких миллионов лет до нескольких недель или часов в зависимости от мощности вычислительных ресурсов. Первые такие попытки были предприняты в 1954 году Нильсом Баричелли. Они увенчались опубликованием результатов его исследований в том же году. В 1957 году австралийский генетик Алекс Фразер подхватил эстафету и опубликовал серию работ, связанную с моделированием искуственного отбора среди организмов с множественным контролем измеримых характеристик. И это лишь те не многие, но безусловно важные работы, положившие начало тому, что в итоге назвали генетическими алгоритмами.
\chapter{Генетические алгоритмы: понятийный аппарат, принцип работы}
\section{Постановка задачи}
Имеются:
целевая функция -  $f(x)$
множество, на котором задана функция - 
Ставится оптимизационная задача - найти такой элемент (), для которого выполняется одно из следующих неравенств
(задача поиска минимума)
(задача поиска максимума)
В дальнейшем будем считать, что стоит задача поиска минимума.

\section{Описание алгоритма}
Генетические алгоритмы (далее ГА) относятся к стохастическим методам. ГА представляет собой адаптивный поисковый метод, основанный на селекции лучших элементов в популяции, подобно эволюционной теории Ч. Дарвина.

Введем основные понятия, применяемые в ГА:
\begin{itemize}
	\item Вектор - упорядоченный набор чисел.
	\item Хромосома - вектор (или строка) из каких-либо чисел. Каждая позиция хромосомы называется геном.
	\item Особь - вариант решения задачи, состоящий из 1 хромосомы.
	\item Кроссинговер - операция, при которой две хромосомы обмениваются своими частями.
	\item Мутация - случайное изменение одной или нескольких позиций в хромосоме.
	\item Популяция - совокупность особей.
	\item Пригодность (приспособленность) - функция, экстремум которой следует найти.
	\item Эволюция популяции - чередование поколений, в которых хромосомы изменяют свои значения так, чтобы каждое новое поколение наилучшим образом приспасабливалось к внешней среде.
\end{itemsize}

Для начала работы ГА выбирается множество случайных решений (особей) для начальной популяции. Далее приступаем к процессу размножения: попробуем на основе исходной популяции создать новую, так чтобы пробные решения в новой популяции были бы ближе к искомому экстремуму целевой функции. Критерием приспособленности особи является значение целевой функции: чем оно меньше, тем более приспособленной является особь (при поиске минимума). Следующим шагом в работе ГА являются мутации. Если скрещивание приводит к относительно небольшим изменениям пробных решений, то мутации могут привести к существенным изменениям значений пробных решений. После мутаций необходимо сформировать новую популяцию. Подобные действия повторяются итеративно, тем самым моделируется \textquotedblrightэволюционный процесс\textquotedblright. Через несколько поколений мы получим популяцию из похожих и наиболее приспособленных особей. Значение приспособленности наиболее "хорошей" и будет являться решением нашей задачи.

Таким образом генетический алгоритм работает по следующей схеме:
\begin{enumrate}
	\item Генерируем начальную популяцию из N хромосом, размер которой не будет меняться на протяжении всего алгоритма
	\item Выбираем пары хромосом-родителей оператором выбора родителя.
	\item Проводим скрещивание каждой пары оператором скрещивания, производя 2 потомков.
	\item Проводим мутацию потомков оператором мутации.
	\item Формируем новую популяцию оператором отбора.
	\item Повторяем шаги 2-5 пока не будет достигнут критерий окончания процесса.
\end{enumrate}

Критерием окончания алгоритма может являться заданное количество поколений или схождение популяции.
При схождении популяции все особи почти одинаковы и находятся в области некоторого экстремума. Скрещивание практически не изменяет популяцию, а вышедшие из области за счет мутации особи склонны вымирать, так как имеют меньшую приспособленность. Таким образом, схождение популяции обычно означает, что найдено лучшее или близкое к нему решение.

\section{Операторы выбора родителей}
Существует несколько подходов к выбору родительских пар. Наиболее распространнеными являются следующие.
\begin{itemize}[label=$\ast$]
	\item Панмиксия
При данном методе выбора родителей каждому члену популяции случайным образом ставится в соответствие целое число на отрезке $[1; N]$, где $N$ - количество особей в популяции. Данное число рассматривается как номер особи, которая примет участие в скрещивании. При таком выборе некоторые особи не будут участвовать в скрещивании, так как попадут в пару с самими собой. А некоторые особи могут участвовать в нескольких скрещиваниях одновременно.
	\item Инбридинг
При данном методе первый родитель выбирается случайным образом, а вторым родителем является член популяции, ближайший к первому. Под \textquotedblrightближайшим\textquotedblright может пониматься в смысле минимального евклидова расстояния между двумя вещественными векторами.
Инбридинг можно охарактеризовать свойством концентрации поиска в локальных узлах, что приводит к разбиению популяции на группы вокруг подозрительных на экстремумы областей.
	\item Аутбридинг
Данный метод отличается от инбридинга тем, что второй родитель выбирается максимально удаленным от первого.
Аутбридинг направлен на сдерживание сходимости алгоритма к уже найденным решениям, заставляя алгоритм исследовать новые области.
	\item Селекция
При данном методе родителями могут стать лишь те особи, значение приспособленности которых не меньше пороговой величины, например, среднего значения приспособленности по популяции. Такой подход обеспечивает более быструю сходимость алгоритма. Однако для некоторых многомерных задач со сложным ландшафтом целевой функции быстрая сходимость может превратиться в преждевременную сходимость к квазиоптимальному решению. Этот недостаток может быть отчасти скомпенсирован использованием подходящего механизма отбора, который бы \textquotedblrightтормозил\textquotedblright слишком быструю сходимость алгоритма.
	\item Турнирный отбор
Турнирный отбор является вариацией селекции. Из популяции, содержащей $N$ особей, выбирается случайным образом $t$ особей, и лучшая из них записывается в промежуточный массив. Эта операция повторяется $N$ раз. Особи в полученном массиве используются для скрещивания. Размер группы строк, отбираемых для турнира, часто равен 2. Преимуществом данного способа является то, что он не требует дополнительных вычислений.
	\item Рулеточный отбор
Рулеточный способ является вариацией селекции. Особи отбираются с помощью $N$ \textquotedblrightзапусков\textquotedblright рулетки, где $N$ - размер популяции. Колесо рулетки содержит по одному сектору для каждого члена популяции. Размер $i$-го сектора пропорционален вероятности попадания в новую популяцию $P(i)$, вычисляемой по формуле:
\begin{equation}
	P(i) = \frac{f(i)}{\sum{i=1}^{N} f(i)},
\end{equation}
где $f(i)$ - пригодность $i$-й особи.
При таком отборе члены популяции с более высокой приспособленностью с большей вероятностью будут чаще выбираться, чем особи с низкой приспособленностью.
\end{itemize}

\section{Операторы скрещивания}
Оператор скрещивания применяют сразу же после оператора отбора родителей для получения новых особей-потомков.
\begin{itemize}[label=$\ast$]
	\item Дискретная рекомбинация
Дискретная рекомбинация в основном применяется к хромосомам с вещественными генами.
	\begin{itemize}[label=$-$]
		\item Дискретная рекомбинация
Дискретная рекомбинация соответствует обмену генами между особями. Для создания потомков случайно с равной вероятностью выбирается номер одного из двух родителей для каждого гена.
		\item Промежуточная рекомбинация
Промежуточная рекомбинация применима только к вещественным переменным. В данном методе предварительно определяется числовой интервал значений генов потомков, который должен содержать значения генов родителей. Потомки создаются по следующему правилу:
\begin{equation}
	\textrm{Потомок} = textrm{Родитель 1} + \alpha \ast (\textrm{Родитель 2} - \textrm{Родитель 1}),
\end{equation}
где множитель $\alpha$ - случайное число на отрезке $[-d; 1 + d], d \geq 0$
Наиболее оптимальным считается значение $d = 0.25$. Для каждого гена выбирается отдельный множитель. 
	\end{itemize}
	\item Кроссинговер
Рекомбинацию бинарных строк принято называть кроссинговером.
	\begin{itemize}[label=$-$]
		\item Одноточечный кроссинговер
Одноточечный кроссинговер моделируется следующим образом. Случайным образом определяется точка разрыва внутри хромосомы, в которой обе хромосомы делятся на две части и обмениваются ими. Такой тип кроссинговера называется одноточечным, так как при нем родительские хромосомы разделяются только в одной случайной точке.
		\item Двуточечный кроссинговер
При двуточечном кроссинговере выбираются 2 точки разреза. Родительские хромосомы обмениваются сегментами, ограниченными двумя точками. В настоящий момент считается, что двуточечный кроссинговер лучше, чем одноточечный.
		\item Многоточечный кроссинговер
Для многоточечного кроссинговера выбирается случайно без повторений m точек разреза. Для получения двух потомков родительские особи обмениваются случайно выбранными сегментами, ограниченными точками разреза.
	\end{itemize}
\end{itemize}

\section{Операторы мутации}
Оператор мутации необходим для \textquotedblrightвыбивания\textquotedblright популяции из локального экстремума и препятствует преждевременной сходимости. Это достигается за счет того, что изменяется случайно выбранный ген или несколько генов в хромосоме. Для мутации можно выбирать несколько точек из популяции, причем их число может быть случайным.
\begin{itemize}[label=$\ast$]
	\item Плотность мутации
Стратегия мутации с использованием понятия плотности заключается в мутировании каждого гена потомка с заданной вероятностью. Величину вероятности применения мутации к каждому гену выбирают так, чтобы в среднем мутировало от 1 до 10\% генов.
	\item Двоичная мутация
Для особей, кодированных двоичным кодом или кодом Грея, мутация заключается в случайном инвертировании гена.
	\item Другие виды мутации
Для особи, представленной последовательностью генов, можно применить следующие операторы мутации: присоединение случайного гена, вставка случайного гена, удаление случайного гена, обмен местами случайно выбранных генов.
\end{itemize}

\section{Операторы отбора}
Оператор отбора необходим для создания новой популяции.
\begin{itemize}[label=$\ast$]
	\item Отбор усечением
При отборе усечением используют популяции особей-родителей и особей-потомков, отсортированные по возрастанию значений функции пригодности. Особи выбираются в соответствии с порогом $T \in [0;1]$. Порог определяет, какая доля особей, начиная с самой пригодной, будет принимать участие в отборе. Среди особей, попавших "под порог", случайным образом выбирается одна особь и записывается в новую популяцию. Так повторяется $N$ раз, пока размер новой популяции не будет равен старой. Новоя популяция будет состоять из особей с высокой пригодностью, причем некоторые особи могут встречаться несколько раз, а самые пригодные могут и не попасть в популяцию.
	\item Элитарный отбор
При элитарном отборе используют популяции особей-родителей и особей-потомков. В новую популяцию выбираются $N$ самых пригодных особей. Иногда данный метод комбинируют с другим - выбирают в новую популяцию 10\% самых пригодных особей, а остальные 90\% выбирают одним из методов селекции. Иногда эти 90\% создают случайно, как на старте алгоритма. Преимуществом данного метода является то, что допускается потеря лучших решений.
	\item Отбор вытеснением
В отборе вытеснением выбор особи в новую популяцию зависит от величины ее пригодности и от того, есть ли уже в формируемой популяции особь с аналогичным хромосомным набором. Из всех особей с одинаковой приспособленностью предпочтение отдается особям с разными генотипами. При данном методе не только сохраняются лучшие решения, но и поддерживается генетическое разнообразие. Отбор вытеснением наиболее пригоден для многоэкстремальных задач, при этом имеется возможность кроме глобального экстремума выделить локальные экстремумы, значения которых близки к глобальному.
\end{itemize}
\chapter{Модернизация генетических алгоритмов}
При использовании генетических алгоритмов для решения задач оптимизации могут возникать проблемы преждевременной сходимости. То есть можно попасть в локальный оптимум и не выйти из него, в силу того, что мы исчерпали возможности популяции к увеличению разнообразия потомства. Существует несколько модернизаций генетических алгоритмов, дабы избежать той или иной проблемы. Рассмотрим некоторые из них:

\begin{enumerate}
\item При использовании популяции с малым числом особей гены распространяются слишком быстро, то есть особи становятся слишком похожими. Можно решить эту проблему тремя способами:
	 \begin{enumerate}
	\item Увеличение числа особей. Это приведет к использованию дополнительной памяти, но весьма эффективен при использовании на простых функция цели.
	\item Самоадаптация алгоритмов. Это чаще всего используемый подход. Он позволяет использовать малый размер популяции. Идея основана на изменении значения вероятности мутации в зависимости от скрещивающися особей., засчет чего достигается самоуравлемость алгоритма. Такой поход называется динамическими мутациями.
	 \item Создание массива для хранения особей, генотип которых мы утрачиваем во время формирования новых поколений. Это, как и в первом случае, приведет к использованию дополнительной памяти, но позволит добавлять особей, которые могли быть не очень приспособленными ранее, но тех, которые могут дать более приспособленное в новых условиях потомство. Или же позволит увеличить количество плохих генов, чтобы выйти из локального отимума. 
	\end{enumerate}
\item 
\end{enumerate}
\chapter{Преимущества и недостатки генетических алгоритмов}
\chapter{Примеры решения задач}
\section{Задача об оптимальном управлении грузовым составом}
\subsection{Описание задачи}
Люди часто думают, что управлять поездом проще, чем машиной -- руля же нет, всё едет само по себе. Только за сигналами следи да чаёк попивай.
Однако на деле всё оказывается не так радужно.

При ведении грузового состава машинист сталкивается с целым рядом неординарных задач, требующих адекватной оценки ситуации, быстрой реакции, аналитического склада ума и, зачастую, хорошей интуиции.
Давайте обратим внимание на факторы, влияющие на процесс движения грузового состава.

\begin{enumerate}
\item Длина поезда.

Длина грузового состава может достигать нескольких километров. Из-за этого при изменении скорости движения -- торможении и разгоне -- на автосцепки вагонов, находящихся в начале, середине и конце действуют разные силы.

Если действия локомотивной бригады будут опрометчивыми и электровоз начнёт слишком быстро разгоняться, то действующая на передние вагоны сила может оказаться выше критической, и автосцепка порвётся.

Ещё хуже дела обстоят при торможении: воздух в тормозной магистрали распространяется достаточно медленно, выравнивание давления вдоль магистрали может происходить в течение нескольких минут. 

Теперь представим ситуацию: ыпоезд шёл под уклон, затормозил, а дальше начался затяжной подъём. Машинист собирает схему на тягу, локомотив начинает тянуть за собой поезд.

Первые вагоны уже не удерживаются тормозом, чего нельзя сказать про задние. Таким образом мы имеем неиллюзорный шанс порвать автосцепку в конце поезда, а локомотивная бригада скорее всего  попросту не увидит оторвавшийся хвост состава, что чревато самыми неприятными последствиями.

\item Распределение массы вдоль поезда.

Здесь проблема носит тот же характер, что и в предыдущем пункте: если оставить лёгкие вагоны впереди, а тяжёлые сзади, то при резком старте, скорее всего, произойдёт разрыв там, где кончаются пустые и начинаюстя гружёные вагоны. Очень нехорошая ситуация может сложиться на подъёме: пусть, скажем, первая половина поезда порожняя, а вторая гружёная. Тогда машинист может, легко втащив на подъём первую половину вполсилы, добавить позиций, чтобы так же бесхлопотно затащить и вторую. В такой ситуации та же самая сцепка -- на стыке пустых и гружёных вагонов -- получит просто фантастическую нагрузку, ведь её будет в прямом смысле рвать на две части.

\item Погодные условия.

Здесь ситуация схожа с той, которую мы наблюдаем при попытке стронуться с места на завязшем в трясине автомобиле -- момент, подводимый к колесу от двигателя, оказывается больше момента, с которым сила трения покоя действует на колесо, и в результате начинается буксование. На железной дороге буксование можно встретить в куда более простых условиях -- достаточно сильного дождя и слишком тяжёлого поезда.

Но если параметры локомотива на этапе сборки состава подбираются таковыми, чтобы он гарантированно мог стронуть с места поезд в любых погодных условиях, то о торможении уже приходится думать машинисту -- если слишком резко "дать по тормозам"\,, то начнётся буксование и воздух в магистрали очень быстро закончится. Состав останется неуправляемым.
\end{enumerate}

Описав основные проблемные ситуации, мы обнаружили наиболее уязвимые узлы управления:
\begin{enumerate}
\item Автосцепка.

Критических ситуаций, связанных с воздействием на автосцепку, достаточно много, но суть у них одна -- нельзя превышать некоторое \textbf{\textit{пороговое значение}}.

\item Тормоз.
\footnote{\textbf{\textit{Кратко о работе поездного тормоза}} Принцип работы следующий: воздух закачивается компрессором в тормозные резервуары под большим давлением. При необходимости затормозить воздух с задаваемой машинистом интенсивностью вытравливается из резервуара в общую тормозную магистраль, ответвления от которой подведены непосредственно к тормозным колодкам. Соответственно чем выше давление в магистрали, тем сильнее прижимаются колодки к колесу и тем быстрее происходит торможение. При отпуске тормоза воздух из магистрали выпускается в атмосферу. Одновременно включается компрессор и воздух нагнетается в тормозные резервуары заново. В данной задаче важно, что это достаточно длительная процедура}

У поездного тормоза две проблемы -- есть \textbf{\textit{минимальное давление воздуха в магистрали}} и \textbf{\textit{максимальная допустимая сила торможения}}, зависящая от конкретных погодных условий.

\item Ограничения скорости. 

Собственно то, из-за чего приходится разгоняться и тормозить.

\end{enumerate}
\subsection{Постановка задачи}
На вход даются следующие параметры:
\begin{itemize}
\item Максимальная сила тяги, которую способен развить локомотив
\item Предельная нагрузка на автосцепку
\item Погодные условия
\item Предельный коэффициент трения покоя при данных погодных условиях
\item Зависимость силы торможения от давления в магистрали
\item Скорость сброса (набора) воздуха в магистрали в разных положениях тормозного крана
\footnote{Под \textbf{\textit{силой торможения}} будем понимать силу, с которой тормозная колодка прилегает к колесу.}
\item Длина поезда
\item Время прохождения "тормозной волны"\ вдоль одного вагона (время, в течение которого давление ТМ в данном вагоне сравняется с давлением в ТМ соседнего вагона)
\item Распредление массы поезда
\item Профиль пути
\end{itemize}

Требуется предоставить режим движения, при котором будут выполнены следующие условия:
\begin{itemize}
\item Поезд доедет до пункта назначения в целости за наименьшее время
\item На каждый следующий участок пути поезд подходит с максимальным давлением в ТМ и скоростью не выше максимально допустимой
\end{itemize}


Запас топлива считаем неограниченным -- обычно в реальных условиях с этим действительно нет проблем.

Погода в течение всего маршрута следования считается неизменной (ясно, что если погодные условия изменились, можно разбить путь на части, на которых погодные условия постоянны).

Будем считать, что в каждый момент времени поезд целиком находится на участке пути с одним профилем -- с точки зрения анализа опасности ситуации такое приближение недопустимо (мы не можем понять, что будет в ситуации разрыва поезда), но с точки зрения численных расчётов мы не внесём ложных смягчений режима вождения.

\subsection{Формализация}
\subsubsection{Вход}
\begin{itemize}
\item $F_{\text{тяги max}}$ -- предельная сила тяги локомотива, \textit{кН}
\item $F_{\text{СА max}}$ -- предельная нагрузка на автосцепку, \textit{кН}
\item $W$ -- условная единица, характеризующая погодные условия, численно обозначающая степень увлажнённости рельса
\item $\mu_{max}(W)$ -- зависимость предельного коэффициента трения покоя колеса о рельс от погодных условий
\item $F_{br}(P_{\text{ТМ}})$ -- зависимость силы торможения от давления в тормозной магистрали (далее ТМ),

$[F_{br}]=\text{кН}, [P_{\text{ТМ}}]=\text{кПа}$
\item $P_{\text{ТМ max}}, P_{\text{ТМ min}}$ -- максимальное и минимальное допустимые давления в ТМ
\item $V$ -- число вагонов в поезде
\item $l_{\text{в}}$ -- длина вагона, \textit{м}. Полагаем, что все вагоны имеют одинаковую длину
\item $\tau$ -- время распространения тормозной волны вдоль одного вагона,  \textit{с}
\item $v_{\text{ТМ}}(S)=\frac{dP_{ТМ}}{dt}(S)$ -- зависимость скорость стравливания (или набора) воздуха из тормозной магистрали от выбранной машинистом позиции крана S, \textit{кПа/с}, $S=\overline{1,6}$
\item $m(n)$ -- распределение массы поезда от номера вагона, \textit{кт}, $n=\overline{1, V}$
\item
$\{(v_{max}^{(k)}, \alpha^{(k)}, d^{(k)})^{T}\}_{k=\overline{1, M}}$ -- профиль пути. Задаётся в виде трёхкомпонетных векторов, состоящих из предельной скорости на участке (\textit{км/ч}), угла наклона (\textit{радианы}) и длины (\textit{м}).

\end{itemize}

\subsubsection{Выход}

${(\{U^{(k)}\}^{h}, \{ S^{(k)}\}^{h}, h, N)^{T}}_{k \in \mathbf{N}}$ -- кортеж, состоящий из табличных функций (1,  2 компоненты), заданных на  $\{t_j\}^{h}, j=\overline{1,N}: t_i=hi, t_N=T_0$. Первая компонента соответствует доле от максимальной тяги, а вторая -- позиции тормозного крана. Третья и четвёртая компоненты кортежа -- параметры временной сетки, на которой определены функции. \cite{shipTheory}


\subsubsection{Ограничения}

На $k$-ом участке перегона:

$
	\begin{cases}
	\int\displaylimits_{0}^{T_0}{v_{\text{ТМ}}}(t)dt=0 &	(1)\\
	|F_{i}(t)-F_{i-1}(t)| \le F_{\text{СА} max} \forall i = \overline{2,V} \forall t \in \{t^{h}\} &	(2) \\
	v_{k} + \frac{1}{m(1)}\int\displaylimits_{0}^{T_0}F_{1}(t)dt \le v_{k+1} &	(3) \\
	F_{\text{тяги}}(t) \le F_{\text{тяги max}} \forall t \in \{t^{h}\} (\text{временно выкидываем, тяга конст})&	(4) \\
	\int\displaylimits_{0}^{T_0}{v(t)}dt = d^{(k)} &	(5) \\
	F_{\text{тяги}}(t) \cdot v_{\text{ТМ}}(t) \le 0 \forall t \in \{t^{h}\} &	(6)
	\end{cases}
$

(1) -- давление в ТМ не должно измениться к концу перегона

(2) -- в каждый момент времени нагрузка на автосцепку не должна превышать предельную

(3) -- скорость на выходе не должна превышать ограничение на следующем участке. Так как все вагоны движутся с одиноковой скоростью, можно не умаляя общности рассматривать первый вагон

(4) -- тепловоз не может развить мощность, б$\acute{\text{о}}$льшую, чем конструкционная

(5) -- за данный промежуток времени тепловоз должен пройти заданное расстояние

(6) -- в каждый момент либо происходит набор воздуха в ТМ вместе с набором тяги, либо торможение с выключенной тягой

Кроме того, для упрощения задачи будем блокировать ручку тормозного крана до тех пор, пока тормозная волна не дойдёт до конца поезда.


\subsubsection{Функция цели}
$T_{0}\rightarrow min$

\subsubsection{Алгоритм решения}

Мы разобьём общую задачу -- нахождение оптимального режима на всём пути -- на подзадачи нахождения оптимального режима на каждом отдельном участке (будем пренебрегать возможными оптимизационными манёврами на стыках профиля ввиду сложности).

Данную задачу концептуально мы будем решать следующим образом.
В генетический алгоритм будет передаваться желаемое время прохождения перегона $T_0$. Алгоритм будет пытаться найти режим движения, в котором поезд сможет благополучно дойти до пункта назначения за данное время. Параметр $T_0$, исходя из результатов работы генетического алгоритма, будет корректироваться по методу половинного деления. Начальный $T_0$ мы найдём аналитически -- вычислим время, за которое поезд гарантированно сможет преодолеть перегон с указанными выше условиями.

Таким образом, наша задача будет решаться связкой генетического алгоритма и метода половинного деления. За счёт такого подхода мы сможем использовать все достоинства ГА и при этом обойти его недостатки: с помощью ГА мы будем отвечать на вопрос, \textbf{\textit{можно ли}} подобрать режим ведения так, чтобы за данное $T_0$ поезд успешно прошёл участок, а не \textbf{\textit{какое $T_0$ оптимальное}.} Сам параметр $T_0$ же будет вычисляться методом половинного деления, сходимость которого нам точно известна.

Мы интуитивно полагаем, что такой подход лучше, поскольку задавая ГА директивный вопрос: \textbf{\textit{да}} или \textbf{\textit{нет}}, мы, очевидно, имеем более обоснованную надежду на то, что ГА сможет дать корректный ответ, чем задавая вопрос: \textbf{\textit{какое значение оптимально}}, хотя бы потому, что область возможных решений несравнимо меньше.

\section{Задача круглого раскроя}
\subsection{Описание задачи}
Рассматриваемая задача — поиск рационального плана раскроя плоского листа на предметы круглой формы. Задачи такого рода впервые
были поставлены еще в 1940-х академиком Л.В. Канторовичем. С тех пор появилось большое количество новых постановок и методов решения. Однако существуют практически значимые 
постановки задач и технологические ограничения, для которых решение задачи раскроя и
разработка новых методов решения по-прежнему актуальны.
Вообще, задача плоского раскроя — это оптимизационная задача поиска наиболее плотного размещения множества меньших по размеру 
плоских предметов, деталей, на больших объектах, заготовках

\subsection{Математическая постановка задачи}
Заданы следующие параметры и условия:
\begin{itemize}
\item Полубесконечная полоса ширины $W$ 
\item $n$ круглых деталей
\item Радиусы деталей  $r_{i}, i = \overline{1, n}$
\item Детали попарно не персекаются:

$(x_{i} - x_{j})^{2} + (y_{i} - y_{j})^{2} \ge (r_{i} - r_{j})^{2}, i, j = \overline{1, n}, i \neq j$
\item Детали не выходят за границы полосы: 
$%\begin{equation}
\begin{cases}
	x_{i} - r_{i} \ge 0 \\
	y_{i} - r_{i} \le 0,	& i = \overline{1, n} \\
	y_{i} + r_{i} \le W
\end{cases}
$%\end{equation}
\end{itemize}

Здесь ${(x_{i}, y_{i}),   i = \overline{1, n}}$ - координаты центров деталей

Необходимо разместить детали на полосе так, чтобы занимаемая часть полосы была минимальна, т. е. ${\max\limits_{i = \overline{1, n}}(x_{i} + r_{i})\xrightarrow[{(x_i, y_i)}]{} \min}$
\printbibliography

\end{document}
