\documentclass[a4paper, oneside]{book}

\usepackage[utf8]{inputenc}
\usepackage[T1]{fontenc}
\usepackage[T2A]{fontenc}
\usepackage{amsmath}
\usepackage[hmargin={2cm, 2cm}, vmargin={2cm, 2cm}]{geometry}
\usepackage{amssymb}
\usepackage{xcolor}
\usepackage{epstopdf}
\usepackage{titlesec}
\usepackage{indentfirst}

\usepackage{enumitem}

\usepackage{fancyhdr}

\usepackage[russian]{babel}
\usepackage{mathrsfs}
\usepackage{textcomp}
\usepackage{wrapfig}
\usepackage{float}

%biblatex
\bibliographystyle{gost-numeric.bbx}
\usepackage{csquotes}
\usepackage[backend=biber]{biblatex}

\begin{document}
\thispagestyle{empty}

%\begin{center}
%\topskip=8cm
%\end{center}
%\large{}
%\vspace{\stretch{1}}
%\begin{flushright}
%\end{flushright}

\title{Курсовая работа по теме: \\ Генетические алгоритмы}
\author{Выполнили:\\студенты группы 3630102/70201\\Дамаскинский К.\\Колесник В.\\Пестряков Д.\\Рыженко В.}
\maketitle

\pagestyle{fancy}
\fancyhf{}
\fancyfoot[C]{\thepage}
\renewcommand{\headrulewidth}{0pt}
\renewcommand{\footrulewidth}{0pt}
\tableofcontents
\pagebreak

\chapter{Общая концепция генетических алгоритмов}
\chapter{Задача об оптимальном управлении грузовым составом}
\section{Описание задачи}
Люди часто думают, что управлять поездом проще, чем машиной -- руля же нет, всё едет само по себе. Только за сигналами следи да чаёк попивай.
Однако на деле всё оказывается не так радужно.

При ведении грузового состава машинист сталкивается с целым рядом неординарных задач, требующих адекватной оценки ситуации, быстрой реакции, аналитического склада ума и, зачастую, хорошей интуиции.
Давайте обратим внимание на факторы, влияющие на процесс движения грузового состава.

\begin{enumerate}
\item Длина поезда.

Длина грузового состава может достигать нескольких километров. Из-за этого при изменении скорости движения -- торможении и разгоне -- на автосцепки вагонов, находящихся в начале, середине и конце действуют разные силы.

Если действия локомотивной бригады будут опрометчивыми и электровоз начнёт слишком быстро разгоняться, то действующая на передние вагоны сила может оказаться выще критической, и автосцепка порвётся.

Ещё хуже дела обстоят при торможении -- воздух в тормозной магистрали распространяется достаточно медленно, и максимальная нагрузка действует на последние вагоны, так как они по инерции дольше всех остальных двигаются вперёд. Таким образом, локомотивная бригада может попросту не увидеть оторвавшийся хвост состава, что чревато самыми неприятными последствиями, особенно если состав штурмовал затяжной подъём, после которого половина поезда пошла на спуск, а вторая всё ещё движется в гору.

\item Распределение массы вдоль поезда.

Здесь проблема носит тот же характер, что и в предыдущем пункте: если оставить лёгкие вагоны впереди, а тяжёлые сзади, то при резком старте, скорее всего, произойдёт разрыв там, где кончаются пустые и начинаюстя гружёные вагоны. Очень нехорошая ситуация может сложиться на подъёме: пусть, скажем, первая половина поезда порожняя, а вторая гружёная. Тогда машинист может, легко втащив на подъём первую половину, добавить позиций, чтобы так же бесхлопотно затащить и вторую. В такой ситуации та же самая сцепка -- на стыке пустых и гружёных вагонов -- получит просто фантастическую нагрузку, ведь её будет в прямом смысле рвать на две части.

\item{Погодные условия}
Здесь ситуация схожа с той, которую мы наблюдаем при попытке стронуться с места завязшего в трясине автомобиля -- момент, подводимый к колесу, оказывается больше момента, с которым сила трения покоя действует на колесо, и в результате начинается буксование. На железной дороге буксование можно встретить в куда более простых условиях - достаточно сильного дождя и слишком тяжёлого поезда.

Но если параметры локомотива на этапе сборки состава подбираются таковыми, чтобы он гарантированно мог стронуть с места поезд в любых погодных условиях, то о торможении уже приходится думать машинисту -- если слишком резко "дать по тормозам"\,, то начнётся буксование и воздух в магистрали очень быстро закончится. Состав останется неуправляемым.
\end{enumerate}

Описав основные проблемные ситуации, мы обнаружили наиболее уязвимые узлы управления:
\begin{enumerate}
\item Автосцепка.

Критических ситуаций, связанных с воздействием на автосцепку, достаточно много, но суть у них одна -- нельзя превышать некоторое \textbf{\textit{пороговое значение}}.

\item Тормоз.
\footnote{\textbf{\textit{Кратко о работе поездного тормоза}} Принцип работы следующий: воздух закачивается компрессором в тормозные резервуары под большим давлением. При необходимости затормозить воздух с задаваемой машинистом интенсивностью вытравливается из резервуара в общую тормозную магистраль, ответвления от которой подведены непосредственно к тормозным колодкам. Соответственно чем выше давление в магистрали, тем сильнее прижимаются колодки к колесу и тем быстрее происходит торможение. При отпуске тормоза воздух из магистрали выпускается в атмосферу. Одновременно включается компрессор и воздух нагнетается в тормозные резервуары заново. В данной задаче важно, что это достаточно длительная процедура}


У поездного тормоза две проблемы -- есть \textbf{\textit{минимальное давление воздуха в магистрали}} и \textbf{\textit{максимальная допустимая сила торможения}}, зависящая от конкретных погодных условий.

\end{enumerate}
\section{Постановка задачи}
На вход даются следующие параметры:
\begin{itemize}
\item Погодные условия
\item Предельная нагрузка на автосцепку
\item Предельный коэффициент трения покоя при данных погодных условиях
\item Зависимость скорости торможения от давления в магистрали
\item Длина поезда
\item Масса поезда
\item Профиль пути
\end{itemize}
\section{Формализация}

\end{document}