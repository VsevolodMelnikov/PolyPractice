\documentclass[a4paper, oneside]{book}

\usepackage[utf8]{inputenc}
\usepackage[T1]{fontenc}
\usepackage[T2A]{fontenc}
\usepackage{amsmath}
\usepackage[hmargin={2cm, 2cm}, vmargin={2cm, 2cm}]{geometry}
\usepackage{amssymb}
\usepackage{xcolor}
\usepackage{epstopdf}
\usepackage{titlesec}
\usepackage{indentfirst}

\usepackage{enumitem}

\usepackage{fancyhdr}

\usepackage[russian]{babel}
\usepackage{mathrsfs}
\usepackage{textcomp}
\usepackage{wrapfig}
\usepackage{float}

%biblatex
\bibliographystyle{gost-numeric.bbx}
\usepackage{csquotes}
\usepackage[backend=biber]{biblatex}

\addbibresource{literature.bib}
\begin{document}
\thispagestyle{empty}

%\begin{center}
%\topskip=8cm
%\end{center}
%\large{}
%\vspace{\stretch{1}}
%\begin{flushright}
%\end{flushright}

\title{Курсовая работа по теме: \\ Генетические алгоритмы}
\author{Выполнили:\\студенты группы 3630102/70201\\Дамаскинский К.\\Колесник В.\\Пестряков Д.\\Рыженко В.}
\maketitle

\pagestyle{fancy}
\fancyhf{}
\fancyfoot[C]{\thepage}
\renewcommand{\headrulewidth}{0pt}
\renewcommand{\footrulewidth}{0pt}
\tableofcontents
\pagebreak

\chapter{Генетические алгоритмы: понятийный аппарат, принцип работы}
\chapter{Модернизация генетических алгоритмов (ДАНЯ ЧТО ЭТО НАДО ПЕРЕДЕЛАТЬ НАЗВАНИЕ)}
\chapter{Преимущества и недостатки генетических алгоритмов}
\chapter{Примеры решения задач}
\section{Задача об оптимальном управлении грузовым составом}
\subsection{Описание задачи}
Люди часто думают, что управлять поездом проще, чем машиной -- руля же нет, всё едет само по себе. Только за сигналами следи да чаёк попивай.
Однако на деле всё оказывается не так радужно.

При ведении грузового состава машинист сталкивается с целым рядом неординарных задач, требующих адекватной оценки ситуации, быстрой реакции, аналитического склада ума и, зачастую, хорошей интуиции.
Давайте обратим внимание на факторы, влияющие на процесс движения грузового состава.

\begin{enumerate}
\item Длина поезда.

Длина грузового состава может достигать нескольких километров. Из-за этого при изменении скорости движения -- торможении и разгоне -- на автосцепки вагонов, находящихся в начале, середине и конце действуют разные силы.

Если действия локомотивной бригады будут опрометчивыми и электровоз начнёт слишком быстро разгоняться, то действующая на передние вагоны сила может оказаться выше критической, и автосцепка порвётся.

Ещё хуже дела обстоят при торможении: воздух в тормозной магистрали распространяется достаточно медленно, выравнивание давления вдоль магистрали может происходить в течение нескольких минут. 

Теперь представим ситуацию: ыпоезд шёл под уклон, затормозил, а дальше начался затяжной подъём. Машинист собирает схему на тягу, локомотив начинает тянуть за собой поезд.

Первые вагоны уже не удерживаются тормозом, чего нельзя сказать про задние. Таким образом мы имеем неиллюзорный шанс порвать автосцепку в конце поезда, а локомотивная бригада скорее всего  попросту не увидит оторвавшийся хвост состава, что чревато самыми неприятными последствиями.

\item Распределение массы вдоль поезда.

Здесь проблема носит тот же характер, что и в предыдущем пункте: если оставить лёгкие вагоны впереди, а тяжёлые сзади, то при резком старте, скорее всего, произойдёт разрыв там, где кончаются пустые и начинаюстя гружёные вагоны. Очень нехорошая ситуация может сложиться на подъёме: пусть, скажем, первая половина поезда порожняя, а вторая гружёная. Тогда машинист может, легко втащив на подъём первую половину вполсилы, добавить позиций, чтобы так же бесхлопотно затащить и вторую. В такой ситуации та же самая сцепка -- на стыке пустых и гружёных вагонов -- получит просто фантастическую нагрузку, ведь её будет в прямом смысле рвать на две части.

\item Погодные условия.

Здесь ситуация схожа с той, которую мы наблюдаем при попытке стронуться с места на завязшем в трясине автомобиле -- момент, подводимый к колесу от двигателя, оказывается больше момента, с которым сила трения покоя действует на колесо, и в результате начинается буксование. На железной дороге буксование можно встретить в куда более простых условиях -- достаточно сильного дождя и слишком тяжёлого поезда.

Но если параметры локомотива на этапе сборки состава подбираются таковыми, чтобы он гарантированно мог стронуть с места поезд в любых погодных условиях, то о торможении уже приходится думать машинисту -- если слишком резко "дать по тормозам"\,, то начнётся буксование и воздух в магистрали очень быстро закончится. Состав останется неуправляемым.
\end{enumerate}

Описав основные проблемные ситуации, мы обнаружили наиболее уязвимые узлы управления:
\begin{enumerate}
\item Автосцепка.

Критических ситуаций, связанных с воздействием на автосцепку, достаточно много, но суть у них одна -- нельзя превышать некоторое \textbf{\textit{пороговое значение}}.

\item Тормоз.
\footnote{\textbf{\textit{Кратко о работе поездного тормоза}} Принцип работы следующий: воздух закачивается компрессором в тормозные резервуары под большим давлением. При необходимости затормозить воздух с задаваемой машинистом интенсивностью вытравливается из резервуара в общую тормозную магистраль, ответвления от которой подведены непосредственно к тормозным колодкам. Соответственно чем выше давление в магистрали, тем сильнее прижимаются колодки к колесу и тем быстрее происходит торможение. При отпуске тормоза воздух из магистрали выпускается в атмосферу. Одновременно включается компрессор и воздух нагнетается в тормозные резервуары заново. В данной задаче важно, что это достаточно длительная процедура}

У поездного тормоза две проблемы -- есть \textbf{\textit{минимальное давление воздуха в магистрали}} и \textbf{\textit{максимальная допустимая сила торможения}}, зависящая от конкретных погодных условий.

\item Ограничения скорости. 

Собственно то, из-за чего приходится разгоняться и тормозить.

\end{enumerate}
\subsection{Постановка задачи}
На вход даются следующие параметры:
\begin{itemize}
\item Предельная нагрузка на автосцепку
\item Погодные условия
\item Предельный коэффициент трения покоя при данных погодных условиях
\item Зависимость силы торможения от давления в магистрали
\footnote{Под \textbf{\textit{силой торможения}} будем понимать силу, с которой тормозная колодка прилегает к колесу.}
\item Длина поезда
\item Время прохождения "тормозной волны"\ вдоль одного вагона (время, в течение которого давление ТМ в данном вагоне сравняется с давлением в ТМ соседнего вагона)
\item Распредление массы поезда
\item Профиль пути
\end{itemize}

Требуется предоставить режим движения, при котором поезд доедет до пункта назначения в целости за наименьшее время.

Запас топлива считаем неограниченным -- обычно в реальных условиях с этим действительно нет проблем.

Погода в течение всего маршрута следования считается неизменной (ясно, что если погодные условия изменились, можно разбить путь на части, на которых погодные условия постоянны).

\subsection{Формализация}
\textbf{Вход}
\begin{itemize}
\item $F_{\text{СА max}}$ -- предельная нагрузка на автосцепку, \textit{кН}
\item $W$ -- условная единица, характеризующая погодные условия, численно обозначающая степень увлажнённости рельса
\item $\mu_{max}(W)$ -- функциональная зависимость предельного коэффициента трения покоя колеса о рельс от погодных условий
\item $F_{br}(P_{\text{ТМ}})$ -- зависимость силы торможения от давления в тормозной магистрали (далее ТМ),

$[F_{br}]=\text{кН}, [P_{\text{ТМ}}]=\text{кПа}$
\item $N$ -- число вагонов в поезде
\item $l_{\text{в}}$ -- длина вагона, \textit{м}. Полагаем, что все вагоны имеют одинаковую длину
\item $\tau$ -- время распространения тормозной волны вдоль одного вагона,  \textit{с}
\item $m(n)$ -- распределение массы поезда от номера вагона, \textit{кт}, $n=\overline{1..N}$
\item
$\{(v_{max}^{(k)}, \alpha^{(k)})^{T}\}_{k=\overline{1, M}}$ -- профиль пути. Задаётся в виде двухкомпонетных векторов, состоящих из предельной скорости на участке (\textit{км/ч}) и угла наклона (\textit{радианы}).
\cite{shipTheory}
\end{itemize}


\printbibliography

\end{document}